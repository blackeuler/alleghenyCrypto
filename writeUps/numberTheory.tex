\documentclass[12pt]{article}
\usepackage{amsmath,amssymb,amsthm,verbatim}

%*****************************************************************
%			Dimensions

\setlength{\textheight}{9in}
\setlength{\textwidth}{6.5in}
\setlength{\oddsidemargin}{0in}
\setlength{\evensidemargin}{0in}
\setlength{\hoffset}{0in}
\setlength{\voffset}{-1in}
\setlength{\footskip}{.5in}
\setlength{\parskip}{10pt}
\setlength{\parindent}{0pt}




%*****************************************************************

%****************************************************************
              
\pagestyle{empty}    

\begin{document}

\begin{center}
    \textbf{Basic Number Theory and Modulo Joy\\
        Chapter 3}
\end{center}
\begin{enumerate}
    \item
          \begin{enumerate}

              \item Find integers $x$ and $y$ such that $17x+101y =1$\\
                    The way that you start this problem is using the Extended Euclidean Algorithim. You can go about that as follows.
                    You can use the Extended Eucleadean algorithm to find s and t of the following equation \[
                        as+nt = gcd(a,n)
                    \].
                    So in our case $a = 17 ,n =101, gcd(a,n)=1 $.
                    The first step is to do the Euclidean algorithm in order to find the quotients.
                    \begin{align*}
                        101 & = 17\cdot 5 + 16 \\
                        17  & = 16\cdot 1 +1   \\
                        16  & = 1\cdot 16
                    \end{align*}
                    Our quotients are $5,1,16$. The extended Euclidean algorithm gives us a recursion in terms of the quotients $q$ and $s , t$ to find $s_n$ and $t_n$. That is of the following form
                    \begin{align*}
                        x_0 =0, x_1=1,x_j = -q_{j-1}x_{j-1}+x_{j+2} \\
                        y_0 =1, y_1=0,y_j = -q_{j-1}y_{j-1}+y_{j+2}
                    \end{align*}
                    Now knowing this equation we can solve for $x_n$ where $n$ is the last step in the Euclidean Algorithim.
                    \begin{align*}
                         & x_0 = 0 x_1 =1        \\
                         & x_2 = -5 \cdot 1 + 0  \\
                         & x_3 = -1 \cdot -5 + 1
                    \end{align*}
                    Going through these same steps for solving $y_n$ we get $x_n=6$ and $y_n = -1$.
                    Checking our answer we see that this indeed true.
              \item Find $17^{-1} \pmod{101}$\\
                    To find the inverse of $17 \pmod{101}$. We need to find a number $x$ that holds the following condition $17x \equiv 1 \pmod{101}$. If we look close at the equation from part (a). We can write it as the following\begin{align*}
                        17x +101y &=1\\
                        17x&= 1-101y\\
                        17-1&= -101y\\
                        17 - 1& = 101\cdot(-y)\\
                        17x &\equiv 1 \pmod{101}
                    \end{align*}
                    So $x$ is 6 in other words $17^{-1} \pmod{101}$ is 6
          \end{enumerate}
          \newpage
          \item \begin{enumerate}
              \item Solve $7d \equiv 1 \pmod{30}$\\
              Similar to our solution to question 1. We need to use the Euclidean Algorithim to find the quotients then solve for the equation $as+nt =1$ where $a = 7, n=30$. Then $d$ will be $s$.
              \begin{align*}
                30 &= 7 \cdot 4 +2\\
                7 &= 4 \cdot 1 +3\\
                4 &=3 \cdot 1 +1\\
                3 &=1 \cdot 3    
              \end{align*}
              Now knowing this equation we can solve for $x_n$ where $n$ is the last step in the Euclidean Algorithim.
                    \begin{align*}
                         & x_0 = 0, x_1 =1        \\
                         & x_2 = -4 \cdot 1 + 0  \\
                         & x_3 = -1 \cdot -4 + 1\\
                         & x_3 = -1 \cdot 5 + 1
                    \end{align*}
                    Going through these same steps for solving $y_n$ we get $x_n=6$ and $y_n = -1$.
          \end{enumerate}
\end{enumerate}
\end{document}



