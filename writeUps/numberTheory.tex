\documentclass[12pt]{article}
\usepackage{amsmath,amssymb,amsthm,verbatim}

%*****************************************************************
%			Dimensions

\setlength{\textheight}{9in}
\setlength{\textwidth}{6.5in}
\setlength{\oddsidemargin}{0in}
\setlength{\evensidemargin}{0in}
\setlength{\hoffset}{0in}
\setlength{\voffset}{-1in}
\setlength{\footskip}{.5in}
\setlength{\parskip}{10pt}
\setlength{\parindent}{0pt}




%*****************************************************************

%****************************************************************
              
\pagestyle{empty}    

\begin{document}

\begin{center}
    \textbf{Basic Number Theory and Modulo Joy\\
        Chapter 3}
\end{center}
\begin{enumerate}
    \item
          \begin{enumerate}

              \item Find integers $x$ and $y$ such that $17x+101y =1$\\
                    The way that you start this problem is using the Extended Euclidean Algorithim. You can go about that as follows.
                    You can use the Extended Eucleadean algorithm to find s and t of the following equation \[
                        as+nt = gcd(a,n)
                    \].
                    So in our case $a = 17 ,n =101, gcd(a,n)=1 $.
                    The first step is to do the Euclidean algorithm in order to find the quotients.
                    \begin{align*}
                        101 & = 17\cdot 5 + 16 \\
                        17  & = 16\cdot 1 +1   \\
                        16  & = 1\cdot 16
                    \end{align*}
                    Our quotients are $5,1,16$. The extended Euclidean algorithm gives us a recursion in terms of the quotients $q$ and $s , t$ to find $s_n$ and $t_n$. That is of the following form
                    \begin{align*}
                        x_0 =0, x_1=1,x_j = -q_{j-1}x_{j-1}+x_{j+2} \\
                        y_0 =1, y_1=0,y_j = -q_{j-1}y_{j-1}+y_{j+2}
                    \end{align*}
                    Now knowing this equation we can solve for $x_n$ where $n$ is the last step in the Euclidean Algorithim.
                    \begin{align*}
                         & x_0 = 0 x_1 =1        \\
                         & x_2 = -5 \cdot 1 + 0  \\
                         & x_3 = -1 \cdot -5 + 1
                    \end{align*}
                    Going through these same steps for solving $y_n$ we get $x_n=6$ and $y_n = -1$.
                    Checking our answer we see that this indeed true.
              \item Find $17^{-1} \pmod{101}$\\
                    To find the inverse of $17 \pmod{101}$. We need to find a number $x$ that holds the following condition $17x \equiv 1 \pmod{101}$. If we look close at the equation from part (a). We can write it as the following\begin{align*}
                        17x +101y & =1                  \\
                        17x       & = 1-101y            \\
                        17-1      & = -101y             \\
                        17 - 1    & = 101\cdot(-y)      \\
                        17x       & \equiv 1 \pmod{101}
                    \end{align*}
                    So $x$ is 6 in other words $17^{-1} \pmod{101}$ is 6
          \end{enumerate}
          \newpage
    \item \begin{enumerate}
              \item Solve $7d \equiv 1 \pmod{30}$\\
                    Similar to our solution to question 1. We need to use the Euclidean Algorithim to find the quotients then solve for the equation $as+nt =1$ where $a = 7, n=30$. Then $d$ will be $s$.
                    \begin{align*}
                        30 & = 7 \cdot 4 +2 \\
                        7  & = 3 \cdot 2 +1 \\
                        3  & =1 \cdot 3
                    \end{align*}
                    Now knowing this equation we can solve for $x_n$ where $n$ is the last step in the Euclidean Algorithim.
                    \begin{align*}
                         & x_0 = 0, x_1 =1       \\
                         & x_2 = -4 \cdot 1 + 0  \\
                         & x_3 = -3 \cdot -4 + 1 \\
                         & x_3 = 13
                    \end{align*}
                    So $x_n = 13$, thus d = 13.

          \end{enumerate}
    \item \begin{enumerate}
              \item Find all solutions of $12x \equiv 28 \pmod{236}$\\
                    First we start by seeing if the gcd(12,236) is 1. The $gcd(12,236) = 4$. So in order to find the solution we must divide the equation by the gcd(12,236) if 4 doesent divide 28 then there is no solution. If it does then we solve the reduced equation for x. The resulting equation is:$3x \equiv 7 \pmod{59}$. The gcd(3,59) is 1 so we solve like normal.Since the inverse of $3\pmod{59}$ is 20 we multiply both sides by 20. Which gives $x\equiv 140 \pmod{59}$.140 Modulo 59 is 22.So $x \equiv 22,81,140,199,22+59\cdot 5, \dots$
              \item Find all solutions of $12x \equiv 30 \pmod{236}$. Since 30  is not divisible by 4. There is no solution
          \end{enumerate}

    \item[11.] Let p be prime. Show that $a^p \equiv a \pmod{p}$ for all a.
          We can show this is true by using Fermat's Theorem. If we have $a^{p-1} \equiv 1 \pmod{p}$. We can multiply by $a$ on both sides to get the formula. $a^p \equiv a \pmod{p}$. This holds for all a.
    \item[12.] Divide $2^{10203}$ by 101. What is the remainder?\\
          The question is asking for the remainder after dividing a number which is the same as asking for $2^{10203} \pmod{101}$
          First notice that $2^{10203} = 2^{100^{102}} \cdot 2^3$. So we can use Fermats Theorem to simplify $2^{100} \equiv 1 \pmod{101}$. This gives us $1^{102} \cdot 2^3 \pmod{101}$. Which is equal to 8.
    \item[13.] Find the last 2 digits of $123^{562}$.
          Since we are looking for the last 2 digits of a number that is the same as getting the remainder of dividing by 100.Since 100 is composite we can use Eulers $\phi$ equation.\[
            \phi(100) = 100 \cdot (1-\frac{1}{2})(1- \frac{1}{5}) = 100 \cdot \frac{1}{2} \cdot \frac{4}{5}= 100 \cdot \frac{4}{10} = 40 \\
          \].So we know that $123^{40} \equiv \pmod{100}\\$ Using this we can conclude \[
                123^{40 \cdot 14}\cdot 123^2 \equiv 1^{14} \cdot 123^2   \equiv 15129 \equiv 29 \pmod{100}
          \]. Thus the last two digits are 29.
          
\end{enumerate}
\end{document}



